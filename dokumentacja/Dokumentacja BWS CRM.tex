\documentclass[12pt,a4paper]{article}
\usepackage[polish]{babel}
\usepackage[T1]{fontenc}
\usepackage[utf8x]{inputenc}
% \usepackage{hyperref}
\usepackage{url}
\usepackage{graphicx}

\addtolength{\hoffset}{-1.5cm}
\addtolength{\marginparwidth}{-1.5cm}
\addtolength{\textwidth}{3cm}
\addtolength{\voffset}{-1cm}
\addtolength{\textheight}{2.5cm}
\setlength{\topmargin}{0cm}
\setlength{\headheight}{0cm}

\begin{document}

\title{Dokumentacja projektu ZPI}
\author{Gabriela Zborowska}
\author{Kamil Wojnarowski}
\date{\today}

\maketitle

\begin{abstract}
Zarys dokumentacji projektowej stanowiący podstawę do realizacji projektu zespołowego. (Z wykorzystaniem schematyzmu struktury znaczników \LaTeX  stworzonego przez R. J. Wysockiego.)
\end{abstract}

\newpage

\tableofcontents
\listoftables
\listoffigures

\newpage

\section{Tytuł}
\section{Nazwa robocza}
\section{Cel}
Jednoznacznie określony stan w przyszłości, nie mylić z produktem końcowym
\section{Zakres}
\subsection{Analiza wymagań}
 (oraz ,,deasemblacja'' procesu osiągnięcia celu)
\subsection{Wymagania funkcjonalne i niefunkcjonalne}
\subsection{Diagram przypadków użycia i diagram przepływu (opcjonalny)}
\subsection{Dobór technologii}
\section{Scenariusze}
(tytuł, numer, aktorzy, stan wejścia (warunki + dane), przebieg scenariusza, wynik, scenariusz alternatywny, jeśli istnieje)
\section{Estymacja czasowa }
(poszczególnych zadań jak i określenie wymagań MVP oraz terminu końcowego oddania)
\section{Implementacja}
\section{Testy i ich wyniki}
\section{Podsumowanie i bilans}
(MVP vs rzeczywistość)
\newpage
\section{Przykłady użycia elementów języka \LaTeX - nie wchodzi w zakres oddawanej dokumentacji stanowi jedynie przykład}
\begin{large}Powiększona czcionka.\end{large}
{\large To też jest powiększona czcionka.}

Jakiś nowy akapit.



To jest dobór technologii.
"Tekst w cudzysłowie podwójnym maszynowym" (wygląda nienaturalnie).
,,Tekst w cudzysłowie podwójnym''. ``Angielski cudzysłów''.

,,Twardą'' spację oznacza się znakiem tylda \~{} ($\sim$).
Mamy do dyspozycji trzy rodzaje myślników - ,,krótki'', -- ,,normalny'' i --- ,,długi''.

\label{nazwa_etykiety}

Przygotuj stronę w \textbf{HTML}'u, która jest ogłoszeniem o seminarium.
W lewym, górnym rogu strony umieść logo Wydziału Fizyki UW.
Podaj nazwę seminarium (np.~Seminarium Kosmologia i Fizyka Cząstek),
tytuł, imię i nazwisko  wygłaszającego seminarium,
instytucję której jest pracownikiem,
adres, numer sali, datę, godzinę.
 Dodaj także w punktach streszczenie wystąpienia\footnote{Jakaś informacja na marginesie.}.
 W zależności od stopnia ważności informacji, zróżnicuj rodzaj,
wielkość i typ czcionki -- Sekcja \ref{nazwa_etykiety}.

\section{Otoczenia}

\subsection{Formatowanie}

Żyjący w V wieku p.n.e.~prorok Malachiasz był autorem Księgi Malachiasza, będącej
ostatnią w grupie dwunastu ksiąg proroków mniejszych Starego Testamentu. Malachiasz jest
świętym Kościoła katolickiego i Cerkwi prawosławnej. Wśród badaczy nie ma zgody co do
tego, czy słowo Malachiasz to imię, tytuł proroka, czy też przypisana sobie przez
anonimowego autora rola posłańca Bożego. Podobne słowo w takim kontekście pojawia
się w Ml 2,7 i Ml 3,1. Rozbieżności mogły powstać za sprawą greckiego tłumacza, który
w Septuagincie przełożył z hebrajskiego Brzemię słowa Pańskiego w ręce Malachi na
(...) w ręce anioła tj. posła Jego, pozbawiając je jednocześnie cech
imienia własnego. Na tej podstawie Orygenes i Tertulian sądzili, że prorok był aniołem.

\begin{center}
Żyjący w V wieku p.n.e. prorok Malachiasz był autorem Księgi Malachiasza, będącej
ostatnią w grupie dwunastu ksiąg proroków mniejszych Starego Testamentu. Malachiasz jest
świętym Kościoła katolickiego i Cerkwi prawosławnej. Wśród badaczy nie ma zgody co do
tego, czy słowo Malachiasz to imię, tytuł proroka, czy też przypisana sobie przez
anonimowego autora rola posłańca Bożego. Podobne słowo w takim kontekście pojawia
się w Ml 2,7 i Ml 3,1. Rozbieżności mogły powstać za sprawą greckiego tłumacza, który
w Septuagincie przełożył z hebrajskiego Brzemię słowa Pańskiego w ręce Malachi na
(...) w ręce anioła tj. posła Jego, pozbawiając je jednocześnie cech
imienia własnego. Na tej podstawie Orygenes i Tertulian sądzili, że prorok był aniołem.
\end{center}

\begin{flushleft}
Żyjący w V wieku p.n.e. prorok Malachiasz był autorem Księgi Malachiasza, będącej
ostatnią w grupie dwunastu ksiąg proroków mniejszych Starego Testamentu. Malachiasz jest
świętym Kościoła katolickiego i Cerkwi prawosławnej. Wśród badaczy nie ma zgody co do
tego, czy słowo Malachiasz to imię, tytuł proroka, czy też przypisana sobie przez
anonimowego autora rola posłańca Bożego. Podobne słowo w takim kontekście pojawia
się w Ml 2,7 i Ml 3,1. Rozbieżności mogły powstać za sprawą greckiego tłumacza, który
w Septuagincie przełożył z hebrajskiego Brzemię słowa Pańskiego w ręce Malachi na
(...) w ręce anioła tj. posła Jego, pozbawiając je jednocześnie cech
imienia własnego. Na tej podstawie Orygenes i Tertulian sądzili, że prorok był aniołem.
\end{flushleft}

\begin{quote}
Żyjący w V wieku p.n.e.~prorok Malachiasz był autorem Księgi Malachiasza, będącej
ostatnią w grupie dwunastu ksiąg proroków mniejszych Starego Testamentu. Malachiasz jest
świętym Kościoła katolickiego i Cerkwi prawosławnej.

Wśród badaczy nie ma zgody co do
tego, czy słowo Malachiasz to imię, tytuł proroka, czy też przypisana sobie przez
anonimowego autora rola posłańca Bożego. Podobne słowo w takim kontekście pojawia
się w Ml 2,7 i Ml 3,1. Rozbieżności mogły powstać za sprawą greckiego tłumacza, który
w Septuagincie przełożył z hebrajskiego Brzemię słowa Pańskiego w ręce Malachi na
(...) w ręce anioła tj. posła Jego, pozbawiając je jednocześnie cech
imienia własnego. Na tej podstawie Orygenes i Tertulian sądzili, że prorok był aniołem.
\end{quote}

\begin{quotation}
Żyjący w V wieku p.n.e.~prorok Malachiasz był autorem Księgi Malachiasza, będącej
ostatnią w grupie dwunastu ksiąg proroków mniejszych Starego Testamentu. Malachiasz jest
świętym Kościoła katolickiego i Cerkwi prawosławnej.

Wśród badaczy nie ma zgody co do
tego, czy słowo Malachiasz to imię, tytuł proroka, czy też przypisana sobie przez
anonimowego autora rola posłańca Bożego. Podobne słowo w takim kontekście pojawia
się w Ml 2,7 i Ml 3,1. Rozbieżności mogły powstać za sprawą greckiego tłumacza, który
w Septuagincie przełożył z hebrajskiego Brzemię słowa Pańskiego w ręce Malachi na
(...) w ręce anioła tj. posła Jego, pozbawiając je jednocześnie cech
imienia własnego. Na tej podstawie Orygenes i Tertulian sądzili, że prorok był aniołem.
\end{quotation}

{\footnotesize
\begin{verbatim}
<html>
<head>
  <meta http-equiv="Content-type" content="text/html; charset=UTF-8" />
  <meta http-equiv="Content-language" content="pl" />
  <link href='style.css' rel='stylesheet' type='text/css' />
  <title>R. J. Wysocki</title>
</head>
\end{verbatim}
}

\subsection{Wypunktowanie i numeracja}

\begin{itemize}
\item[--] Żyjący w V wieku p.n.e.~prorok Malachiasz był autorem Księgi Malachiasza, będącej
  ostatnią w grupie dwunastu ksiąg proroków mniejszych Starego Testamentu. Malachiasz jest
  świętym Kościoła katolickiego i Cerkwi prawosławnej.
\item Drugi punkt.
  \begin{itemize}
  \item Pierwszy podpunkt.
  \item Drugi podpunkt.
  \end{itemize}
\item Trzeci punkt.
\item Czwarty punkt.
\end{itemize}

\begin{enumerate}
\item Żyjący w V wieku p.n.e.~prorok Malachiasz był autorem Księgi Malachiasza, będącej
  ostatnią w grupie dwunastu ksiąg proroków mniejszych Starego Testamentu. Malachiasz jest
  świętym Kościoła katolickiego i Cerkwi prawosławnej.
\item Drugi punkt.
  \begin{itemize}
  \item[--] Pierwszy podpunkt.
  \item[--] Drugi podpunkt.
  \end{itemize}
\item Trzeci punkt.
  \begin{enumerate}
  \item Pierwszy podpunkt.
  \item Drugi podpunkt..
    \begin{enumerate}
    \item Ala
    \item ma
    \item kota.
    \end{enumerate}
  \end{enumerate}
\item Czwarty punkt.
\end{enumerate}

\begin{description}
\item[nazwa 1] -- opis nazwy 1.
\item[nazwa 2] -- opis nazwy 2.
\item[nazwa 3] -- opis nazwy 3.  Opis może być dłuższy, niż jeden wiersz i warto
  zobaczyć co się wtedy stanie.
\end{description}

\subsection{Tabele}

\begin{table}[htb]
  \begin{tabular}{clr}
  {\bf Wyśrodkowanie} & {\bf Do lewej} & {\bf Do prawej} \\
  Treść & Treść & Treść \\
  Kolejny wiersz & Kolejnuy wiersz & Kolejny wiersz \\
  \end{tabular}
\caption{Tabela}
\label{tab:bez_ramek}
\end{table}

\begin{table}[htb]
  \begin{tabular}{c|l|r}
  {\bf Wyśrodkowanie} & {\bf Do lewej} & {\bf Do prawej} \\
  Treść & Treść & Treść \\
  Kolejny wiersz & Kolejnuy wiersz & Kolejny wiersz \\
  \end{tabular}
\caption{Tabela z liniami pionowymi między kolumnami}
\label{tab:pionowe}
\end{table}

\begin{table}[htb]
  \begin{tabular}{|c|l|r|}
  \hline
  {\bf Wyśrodkowanie} & {\bf Do lewej} & {\bf Do prawej} \\
  \hline
  \hline
  Treść & Treść & Treść \\
  \hline
  Kolejny wiersz & Kolejnuy wiersz & Kolejny wiersz \\
  \hline
  \end{tabular}
\caption{Tabela z liniami pionowymi między kolumnami i poziomymi między wierszami}
\label{tab:ramki}
\end{table}

\begin{table}[htb]
  \begin{tabular}{|c|l|p{6cm}|}
  \hline
  {\bf Wyśrodkowanie} & {\bf Do lewej} & {\bf Paragraf} \\
  \hline
  \hline
  Treść & Treść & Treść \\
  \hline
  Kolejny wiersz & Kolejnuy wiersz &
  Żyjący w V wieku p.n.e.~prorok Malachiasz był autorem Księgi Malachiasza, będącej
  ostatnią w grupie dwunastu ksiąg proroków mniejszych Starego Testamentu. \\
  \hline
  \end{tabular}
\caption{Tabela z dłuższym tekstem}
\label{tab:paragraf}
\end{table}

Na rysunku \ref{sinus} jest przedstawiony wykres funkcji sin(x).  W tablicach
\ref{tab:bez_ramek}, \ref{tab:pionowe}, \ref{tab:ramki}, \ref{tab:paragraf}
mamy przykłady zastosowania środowiska \verb#tabular#.

Odwołanie do literatury -- pierwsza pozycja w spisie \cite{Wikipedia}, moja strona
domowa \cite{RJW}.

\begin{figure}[htb!p]
% \includegraphics[width=0.9\textwidth]{sinus.pdf}
\caption{Wykres funkcji sin(x)}
\label{sinus}
\end{figure}

\begin{thebibliography}{9}
\bibitem{Wikipedia} {\it Pauli matrices}
  (\url{http://en.wikipedia.org/wiki/Pauli_matrices}).
\bibitem{RJW} {\it Moja strona} (\url{http://www.fuw.edu.pl/~rwys}).
\end{thebibliography}

\end{document}